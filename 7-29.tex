% Created 2016-07-29 Fri 15:46
\documentclass[11pt]{article}
\usepackage[utf8]{inputenc}
\usepackage[T1]{fontenc}
\usepackage{fixltx2e}
\usepackage{graphicx}
\usepackage{grffile}
\usepackage{longtable}
\usepackage{wrapfig}
\usepackage{rotating}
\usepackage[normalem]{ulem}
\usepackage{amsmath}
\usepackage{textcomp}
\usepackage{amssymb}
\usepackage{capt-of}
\usepackage{hyperref}
\usepackage{minted}
\author{章立}
\date{\today}
\title{}
\hypersetup{
 pdfauthor={章立},
 pdftitle={},
 pdfkeywords={},
 pdfsubject={},
 pdfcreator={Emacs 24.5.1 (Org mode 8.3.4)}, 
 pdflang={English}}
\begin{document}

\tableofcontents

\section{This week}
\label{sec:orgheadline7}
\subsection{学习docker}
\label{sec:orgheadline1}
\begin{itemize}
\item \href{http://192.168.199.107:10080/w/docker_usage_and_study/}{骆裕龙文档}
\item \href{http://docs.daocloud.io/faq/docker101}{DaoCloud文档}
\item \href{https://docs.docker.com/}{docker官方文档}
\item \href{https://www.youtube.com/playlist?list=PLkA60AVN3hh_6cAz8TUGtkYbJSL2bdZ4h}{youtube视频教程}
\item \href{https://www.youtube.com/user/dockerrun}{dockercon}
\end{itemize}
\subsection{学习tmux}
\label{sec:orgheadline2}
\begin{itemize}
\item 适合长时间ssh操作
\item 可以保存和重建所有的会话状态
\item 网络坏境不稳定时,不会中断当前的操作
\item \href{https://www.youtube.com/playlist?list=PLtK75qxsQaMJ_DmXk9yZbCBJuG9HRwlGc}{tmux视频教程}
\end{itemize}
\subsection{搭建deep learning框架}
\label{sec:orgheadline3}
\begin{itemize}
\item All-in-one Docker image for Deep Learning
\begin{itemize}
\item Ubuntu 14.04
\item CUDA 7.5 (GPU version only)
\item cuDNN v4 (GPU version only)
\item Tensorflow
\item Caffe
\item Theano
\item Keras
\item Lasagne
\item Torch (includes nn, cutorch, cunn and cuDNN bindings)
\item iPython/Jupyter Notebook (including iTorch kernel)
\item Numpy, SciPy, Pandas, Scikit Learn, Matplotlib
\item A few common libraries used for deep learning
\end{itemize}
\end{itemize}
\subsection{安装dl-docker}
\label{sec:orgheadline4}
\begin{enumerate}
\item 安装docker
\item 安装nvidia驱动
\item 安装nvidia-docker
\item 安装dl-docker
\end{enumerate}
\begin{minted}[]{bash}
# Find your graphics card model
lspci | grep -i nvidia

# Don't install two different drivers
sudo apt-get purge bumblebee

# We will install the drivers using apt-get.
sudo add-apt-repository ppa:graphics-drivers/ppa
sudo apt-get update
sudo apt-cache search nvidia
sudo apt-get install nvidia-367

# Don't need to install CUDA


# Install nvidia-docker and nvidia-docker-plugin
wget -P /tmp https://github.com/NVIDIA/nvidia-docker/releases/download/v1.0.0-rc.3/nvidia-docker_1.0.0.rc.3-1_amd64.deb
sudo dpkg -i /tmp/nvidia-docker*.deb && rm /tmp/nvidia-docker*.deb

# Test nvidia-smi
nvidia-docker run --rm nvidia/cuda nvidia-smi

# Install dl-docker
docker pull floydhub/dl-docker:cpu

# CPU Version
docker build -t floydhub/dl-docker:cpu -f Dockerfile.cpu .

# GPU Version
docker build -t floydhub/dl-docker:gpu -f Dockerfile.gpu .
\end{minted}
\subsection{使用和调试deep learning框架caffe}
\label{sec:orgheadline5}
\begin{itemize}
\item 学习中
\end{itemize}
\subsection{筛选简历}
\label{sec:orgheadline6}

\section{Next week}
\label{sec:orgheadline9}
\subsection{学习使用caffe}
\label{sec:orgheadline8}
\end{document}
